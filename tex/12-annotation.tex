\chapter*{\centering Аннотация}


Магистерская диссертация «Спектральная диагностика пылевой плазмы в газовом разряде постоянного тока»
содержит 63 страницы, рисунков~-~21, таблиц~-~1, использовано источников~-~23.

Ключевые слова: пылевая плазма, газовый разряд, неон, низкое давление, спектральная диагностика,
кинетическое уравнение Больцмана, электронная температура, эмиссионная спектроскопия, микрогравитация.

Объект исследования~-~газоразрядная пылевая плазма в условиях невесомости,
получаемая на Российско-европейской научной аппаратуре «Плазменный~кристалл~-~4» (НА «ПК-4»).
Целью работы является разработка нового метода спектроскопической диагностики температуры электронов (ТЭ) в
положительном столбе газового разряда постоянного тока и исследование влияния
протяженного пылевого облака на ТЭ. Актуальность работы
заключается в отсутствии на данный момент независимых диагностических методов
ТЭ для условий данного эксперимента.
Предложена и применена оригинальная методика, основывающаяся на
относительном изменении интенсивностей спектральных линий неона в
присутствии пылевого облака, для реализации которой, было решено
кинетическое уравнение Больцмана. Показано, что в указанных
экспериментальных условиях напряженность осевого электрического поля в облаке
возрастает с $2.2$~В/см до $2.8$~В/см, а ТЭ «хвоста» функции распределения
электронов по энергиям возрастает с $3.2$~эВ до $3.5$~эВ. Данные результаты
получены впервые. Для обработки многочисленных эмиссионных спектров неона
был создан веб-сервис «Spectral Analyzer PK-4», позволяющий в дистанционном
режиме проводить первичную обработку спектров: обзор, усреднение, вычет
темнового тока, коррекцию спектральной чувствительности. Данный
веб-сервис будет рекомендован международной научно-технической группе по НА
«ПК-4» для использования при обработке спектральных данных.

\vfill
\vfill
\begin{minipage}{.49\textwidth}\end{minipage}
\hfill
\begin{minipage}{.49\textwidth}
    Автор: \uline{\hfill} Шоненков А.В.
\end{minipage}
\vfill

\thispagestyle{empty}