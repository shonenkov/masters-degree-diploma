\chapter*{Аннотация}

В данном исследовании разработан метод измерения относительного изменения температуры электронов $T_e$
в стационарном однородном положительном столбе слаботочного газового разряда
постоянного тока в неоне низкого давления под влиянием протяженного пылевого
облака в космических условиях по изменению интенсивности различных спектральных линий.
Был проведен эксперимент, с помощью которого получена оценка согласованности с разработанным подходом, а также
впервые на основе эмиссионных спектров расчитано абсолютное значение электронной температуры
для данного газового разряда при возмущении пылевым облаком, используя параметры невозмущенного разряда:\\[3mm]
\begin{center}
    \begin{tabular}{|l|c|c|}
    \hline
          Разряд   & $E_z$, В/см & $T_e$, эВ \\
    \hline
    невозмущенный  & $ 2.2 $     & $ 3.2 $   \\
    возмущенный    & $ 2.8 $     & $ 3.5 $   \\
    \hline
    \end{tabular}\\[10mm]
\end{center}

В ходе данной работы был создан веб-сервис «Spectral~Analyzer~PK-4», который позволяет обрабатывать
спектральные данные экспериментов, проведенных на научной аппаратуре «Плазменный~кристалл~-~4». Данный веб-сервис
уникален и также имеет свой научный вклад в мировое сообщество.

\vfill
\vfill
\begin{minipage}{.49\textwidth}\end{minipage}
\hfill
\begin{minipage}{.49\textwidth}
    Автор: \uline{\hfill} Шоненков А.В.
\end{minipage}
\vfill
