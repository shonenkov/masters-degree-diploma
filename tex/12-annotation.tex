\chapter*{Аннотация}

В данном исследовании разработан метод измерения относительного изменения температуры электронов $T_e$
в стационарном однородном положительном столбе слаботочного газового разряда
постоянного тока в неоне низкого давления под влиянием протяженного пылевого
облака в космических условиях по изменению интенсивности различных спектральных линий.
Был проведен эксперимент, с помощью которого получена оценка согласованности с разработанным подходом.
Впервые на основе эмиссионных спектров было расчитано абсолютное значение электронной температуры $T_e$ и осевого
электрического поля $E_z$ возмущенного пылевым облаком газового разряда, используя параметры невозмущенного разряда.
Невозмущенный разряд: $E_z = 2.2$~В/см, $T_e = 3.2$~В/см.
Возмущенный разряд: $E_z = 2.8$~В/см, $T_e = 3.5$~В/см.

В ходе данной работы был создан веб-сервис «Spectral~Analyzer~PK-4», который позволяет обрабатывать
спектральные данные экспериментов, проведенных на научной аппаратуре «Плазменный~кристалл~-~4». Данный веб-сервис
уникален и также имеет свой научный вклад в мировое сообщество.

\vfill
\vfill
\begin{minipage}{.49\textwidth}\end{minipage}
\hfill
\begin{minipage}{.49\textwidth}
    Автор: \uline{\hfill} Шоненков А.В.
\end{minipage}
\vfill
