\chapter*{Аннотация}

Проведено исследование влияния протяженного пылевого облака на температуру
электронов однородного положительного столба газового разряда постоянного
тока. Исследование проводилось на борту Международной космической станции в
рамках совместного российско-европейского космического эксперимента
«Плазменный~кристалл~-~4» в газоразрядной трубке с внутренним диаметром $30$~мм
в неоне при давлении $60$~Па и разрядном токе $1$~мА. Пылевое облако создавалось из
монодисперсных пластиковых частиц диаметром $3.38$~мкм, численная
концентрация пылевых частиц в облаке составляла $2\cdot10^4$~см$^{-3}$, а диаметр самого
облака составлял $9$~мм. Для определения изменения температуры электронов в
присутствии пылевого облака предложена и применена оригинальная методика,
основывающаяся на измерении относительного изменения интенсивностей
спектральных линий неона в присутствии пылевого облака. Для интерпретации
полученных результатов функция распределения электронов по энергиям
находилась из решения уравнения Больцмана. Показано, что в указанных
экспериментальных условиях напряженность осевого электрического поля в облаке
возрастает с $2.2$~В/см до $2.8$~В/см, а температура «хвоста» функции распределения
электронов по энергиям возрастает с $3.2$~эВ до $3.5$~эВ. Для обработки
многочисленных эмиссионных спектров неона был создан веб-сервис «Spectral
Analyzer PK4», позволяющий в дистанционном режиме проводить первичную
обработку спектров: обзор, усреднение, вычет темнового ток, проводить
коррекцию спектральной чувствительности.

\vfill
\vfill
\begin{minipage}{.49\textwidth}\end{minipage}
\hfill
\begin{minipage}{.49\textwidth}
    Автор: \uline{\hfill} Шоненков А.В.
\end{minipage}
\vfill
