\chapter*{Аннотация}

Объектом исследования является газоразрядная пылевая плазма в условиях невесомости,
получаемая на российско-европейской научной аппаратуре «Плазменный~кристалл~-~4»
(НА «ПК-4») на борту Международной космической станции (МКС) в рамках
одноименного Российско-европейского проекта. Целью работы является разработка
нового метода спектроскопической диагностики температуры электронов в
положительном столбе газового разряда постоянного тока и исследование влияния
протяженного пылевого облака на температуру электронов. Актуальность работы
заключается в отсутствии на данный момент методов спектроскопической диагностики
электронной температуры адекватных для условий данного эксперимента и его
аппаратурного обеспечения, а условия микрогравитации позволяют создавать значительно
более объемные пылевые облака в плазме разряда и тем самым увеличивать искомый
эффект и делать его доступным для измерений. Исследование проводилось в
газоразрядной трубке с внутренним диаметром $30$~мм в неоне при давлении $60$~Па и
разрядном токе $1$~мА. Пылевое облако создавалось из монодисперсных
пластиковых частиц диаметром $3.38$~мкм, численная концентрация пылевых частиц
в облаке составляла $2\cdot10^4$~см$^{-3}$, а диаметр самого облака составлял $9$~мм. Для
определения изменения температуры электронов в присутствии пылевого облака
предложена и применена оригинальная методика, основывающаяся на измерении
относительного изменения интенсивностей спектральных линий неона в
присутствии пылевого облака. Для интерпретации полученных результатов
функция распределения электронов по энергиям находилась из решения
кинетического уравнения Больцмана. Показано, что в указанных
экспериментальных условиях напряженность осевого электрического поля в облаке
возрастает с $2.2$~В/см до $2.8$~В/см, а температура «хвоста» функции распределения
электронов по энергиям возрастает с $3.2$~эВ до $3.5$~эВ. Данные результаты
получены впервые. Для обработки многочисленных эмиссионных спектров неона
был создан вебсервис «Spectral Analyzer PK4», позволяющий в дистанционном
режиме проводить первичную обработку спектров: обзор, усреднение, вычет
темнового тока, проводить коррекцию спектральной чувствительности. Данный
веб-сервис будет рекомендован международной научно-технической группе по НА
“ПК-4” для использования при обработке спектральных данных.

%Проведено исследование влияния протяженного пылевого облака на температуру
%электронов однородного положительного столба газового разряда постоянного
%тока. Исследование проводилось на борту Международной космической станции в
%рамках совместного российско-европейского космического эксперимента
%«Плазменный~кристалл~-~4» в газоразрядной трубке с внутренним диаметром $30$~мм
%в неоне при давлении $60$~Па и разрядном токе $1$~мА. Пылевое облако создавалось из
%монодисперсных пластиковых частиц диаметром $3.38$~мкм, численная
%концентрация пылевых частиц в облаке составляла $2\cdot10^4$~см$^{-3}$, а диаметр самого
%облака составлял $9$~мм. Для определения изменения температуры электронов в
%присутствии пылевого облака предложена и применена оригинальная методика,
%основывающаяся на измерении относительного изменения интенсивностей
%спектральных линий неона в присутствии пылевого облака. Для интерпретации
%полученных результатов функция распределения электронов по энергиям
%находилась из решения уравнения Больцмана. Показано, что в указанных
%экспериментальных условиях напряженность осевого электрического поля в облаке
%возрастает с $2.2$~В/см до $2.8$~В/см, а температура «хвоста» функции распределения
%электронов по энергиям возрастает с $3.2$~эВ до $3.5$~эВ. Для обработки
%многочисленных эмиссионных спектров неона был создан веб-сервис «Spectral
%Analyzer PK4», позволяющий в дистанционном режиме проводить первичную
%обработку спектров: обзор, усреднение, вычет темнового ток, проводить
%коррекцию спектральной чувствительности.

\vfill
\vfill
\begin{minipage}{.49\textwidth}\end{minipage}
\hfill
\begin{minipage}{.49\textwidth}
    Автор: \uline{\hfill} Шоненков А.В.
\end{minipage}
\vfill
