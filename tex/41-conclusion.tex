\Conclusion % заключение к отчёту

В ходе решения поставленной задачи была разработана методика определения относительного изменения электронной
температуры по эмиссионным спектрам в положительном столе газового разряда постоянного тока, которая показала
достаточно хорошую согласованность с экспериментом. Проведена спектральная диагностика данного газового разряда
в возмущенном и невозмущенном пылевым облаком состояниях, в ходе которой наблюдался неравномерный рост интенсивностей
спектральных линий для различных возбужденных состояний. Данный эффект в космических условиях выражен сильнее, чем
в условиях земной гравитации. Оценены абсолютные значения электронной температуры и осевого электрического поля
для возмущенного пылевыми частицами газового разряда на основе параметров невозмущенного разряда
по эмиссионным спектрам, что не было сделано ранее. Реализован веб-сервис «Spectral~Analyzer~PK-4», который
используется для обработки огромного числа спектральных данных, полученных с научной аппаратуры
«Плазменный~кристалл~-~4».

