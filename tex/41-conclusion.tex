\Conclusion % заключение к отчёту

В соответствии с поставленной научной задачей проведен обзор и отбор
экспериментальных данных Российско-европейского космического эксперимента
«Плазменный~кристалл~-~4». Отобранные экспериментальные данные включают в
себя: видеофайлы с обзорным изображением трубки газового разряда,
контролирующие распределение яркости свечения положительного столба и
положение плазменно-пылевого облака в положительном столбе, видеофайлы
высокого разрешения, характеризующие диаметр пылевого облака и концентрацию
пылевых частиц в нем, ток разряда и давление неона в газоразрядной камере,
сопутствующие спектры излучения положительного столба. Отобранные
эмиссионные спектры были обработаны с помощью специально разработанной
программой «Spectral Analyzer PK-4» с целью получения необходимой величины
отношения сигнал/шум. Получены отношения интенсивностей эмиссионных
спектральных линий неона, излучаемых положительным столбом с плазменно-пылевым
облаком и положительным столбом без пылевого облака. Показано, что
величина этого отношения зависит от энергии верхнего уровня соответствующего
энергетического перехода и варьируется от $1.4$ для уровней с энергий $18.5$~эВ до
$1.65$ для уровней с энергией $20$~эВ. Качественно очевидно, что такое увеличение
населенностей возбужденных уровней происходит вследствие увеличения
электронной температуры. Ввиду очень низкой плотности плазмы, количественный
анализ наблюдаемого эффекта проводился в рамках упрощенной модели заселения
возбужденных состояний прямым электронным ударом. Для определения
скоростей заселения возбужденных состояний функция распределения электронов
по энергиям определялась путем решения кинетического уравнения Больцмана для условий
данного эксперимента. Полученная функция имеет приблизительно «двухтемпературный» вид:
распределение электронов в диапазоне до $16$~эВ описывается
максвелловской функцией с температурой около $7$~эВ, в то время как в диапазоне
свыше $17$~эВ описывается максвелловской функцией с температурой около $3.5$~эВ,
что хорошо согласуется с литературными данными. Показано, что если взять за
исходную температуру «хвоста» электронной функции распределения
литературное значение $3.2$~эВ, то согласно предложенной методике и
спектральным данным в плазменно-пылевом облаке эта температура возрастет до
$3.5$~эВ. Таким образом, продемонстрирована возможность измерения температуры
«хвоста» функции распределения электронов на основе спектральных данных на
научной аппаратуре «Плазменный~кристалл~-~4».