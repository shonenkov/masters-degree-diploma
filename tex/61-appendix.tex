\App
\label{app}

\section{Приложение А}
\label{app:app1}
Пример записи блока (одного спектра) в эксперименте. Спектральный файл обычно содержит более 1000 таких блоков, которые
следуют друг за другом.

\begin{small}
\begin{verbatim}
###############################################################################
# PK4 EAC SW -- Spectrometer
# started   2016-10-12'13:57:45.58 ~
# HPC=7679703032397 / HPCFreq=1496280000 Hz => HPC uptime = 5132.530698 s
# 2016-10-12'13:57:42.55; spectrometer commanded
# 2016-10-12'13:57:45.58; spectrometer response received
#--- spectrum ---
# 65535; spectrum start marker
#     0; data size flag
#     1; nr scans accumulated
#   750; integration time /ms
#     0; reserved value FPGA_ESV_MSW
# 12118; reserved value FPGA_ESV_LSW
#     0; pixel mode
   0:     0
   1:   471
   2:   484
   3:   466
   4:   510
   5:   506
   6:   484
   7:   451
   8:   491
   9:  1729
………..
1022:   537
1023:   623
1024:   557
1025:   493
1026:   525
………..
2038:   506
2039:   578
2040:   585
2041:   548
2042:   618
2043:   546
2044:   650
2045:   549
2046:   541
2047:   516
# 65533; spectrum end marker
#=== spectrum === read-out time = 2.172 s
# ended     2016-10-12'13:57:45.63 ~ (execution time = 0.051892 s)
# PK4 EAC SW -- Spectrometer
###############################################################################
\end{verbatim}
\end{small}

\section{Приложение Б}
\label{app:app2}