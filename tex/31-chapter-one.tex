\chapter{Спектральная диагностика~низкотемпературной плазмы газового разряда}
%\chapter{СПЕКТРАЛЬНАЯ ДИАГНОСТИКА НИЗКОТЕМПЕРАТУРНОЙ ПЛАЗМЫ ГАЗОВОГО РАЗРЯДА}
\label{cha:ch_1}
\section{Кинетика заселения возбужденных атомных состояний в плазме}
Спектры излучения газоразрядной плазмы определяются населенностью \math{N_j}$ соответствующих возбужденных атомных уровней \math{E_j}$.
Тогда интенсивность соответствующей атомной спектральной линии составит:
\begin{equation}
    I_{ji} = A_{ji}h\nu_{ji}⋅N_j
\end{equation}
где \math{A_{ji}}$  - коэффициент Эйнштейна для перехода \math{j → i}$, \math{N_j}$ - населенность возбужденного уровня \math{j}$.

Таким образом, задача спектральной диагностики плазмы сводится к построению теоретических моделей, связывающих
параметры плазмы (в первую очередь, концентрации электронов \math{n_e}$ и их температуру \math{T_e}$) с интенсивностями спектральных
линий \math{I_{ji}}$. Выбор той или иной модели зависит от параметров плазмы: ее химического состава, плотности,
степени ионизации и равновесности. В данной работе экспериментально и теоретически исследуется стационарная сильно
неравновесная плазма положительного столба слаботочного газового разряда постоянного тока (\math{I_{DC}~=~1~мА}$)
в неоне при давлении \math{P~=~50~Па}$, причем степень ионизации плазмы α очень мала (\math{\alpha~\sim~10^{-8}}$).
Выбор этих параметров обуславливается практическим случаем, рассматриваемым в этой работе.
\begin{figure}[t]
    \begin{center}
         \subfloat[\label{sub:fig11a}]{
           \includegraphics[width=0.35\textwidth]{figures/fig11a}
         }
         \hspace{0.05\columnwidth}
         \subfloat[\label{sub:fig11b}]{
           \includegraphics[width=0.305\textwidth]{figures/fig11b}
         }
         \caption{Схема основных процессов уровня: \pt(a) заселение, \pt(b) расселение.}
         \label{fig:fig11}
    \end{center}
\end{figure}

Населенность \math{N_j}$ уровня \math{E_j}$ в стационарном случае определяется балансом процессов его заселения
и расселения. Схема основных процессов заселения уровня \math{E_j}$ представлена на рис.~\ref{fig:fig11}~\subref{sub:fig11a}.
Основным процессом заселения рассматриваемого уровня \math{E_j}$ является его заселение прямым
электронным переходом с основного состояния \math{E_0}$ и с метастабильного \math{E_i}$.

Скорости процессов \math{C_{0j}}$ и \math{C_{ij}}$ определяются соотношениями:
\begin{equation}C_{0j} = \sqrt{2 \over m_e} \int_{E_j}^{\infty} \sigma_{0j}(E) f_e(E) \sqrt{E} dE\end{equation}
и
\begin{equation}C_{ij} = \sqrt{2 \over m_e} \int_{E_j - E_i}^{\infty} \sigma_{ij}(E) f_e(E) \sqrt{E} dE\end{equation}
соответственно, где …

Сечения \math{\sigma_{0j}(E)}$ и \math{\sigma_{ij}(E)}$ расчетные и экспериментальные, можно найти в литературе
или базе данных NIST [ССЫЛКА НА ИСТОЧНИК]. Типичный вид сечений представлен на рис. \ref{fig:fig12}.
Что касается вида ФРЭ, то она сильно зависит от параметров плазмы. При высоких давлениях ФРЭ приближается
к максвелловской функции. Однако, при низких давлениях плазма сильно неравновесна, и вид функции ФРЭ должен
быть определен дополнительными методами. Кроме столкновительного заселения уровень \math{E_j}$ заселяется
также путем радиационного распада верхних \math{k}$-уровней \math{E_k~>~E_j}$ cо скоростью \math{A_{kj}}$ при \math{k > j}$.
Значения \math{A_{kj}}$ также табулированы в базе данных NIST.
\begin{figure}[t]
  \centering
  \includegraphics[width=8cm]{figures/fig12}
  \caption{Типичный вид сечения \math{\sigma}$ и максвелловской ФРЭ. \math{T_e}$ - температура электронов, \math{E_{thr}}$ - пороговая энергия..}
  \label{fig:fig12}
\end{figure}

Схема основных процессов расселения уровня \math{E_j}$ представлена на рис.~\ref{fig:fig11}~\subref{sub:fig11b}.
Этими процессами также являются явления спонтанного распада возбужденных уровней и столкновительные процессы,
индуцированные свободными электронами.

Приравняв скорости заселения и расселения уровня \math{E_j}$, мы получим уравнение относительно ФРЭ \math{f_e(E)}$.
Это уравнение является некорректной задачей и для ее решения необходимы дополнительные данные о виде ФРЭ.
Эти данные могут быть получены путем решения кинетического уравнения Больцмана для соответствующих условий.

\section{Уравнение Больцмана для ФРЭ в положительном столбе газового разряда постоянного тока.}

Плазма тлеющего разряда низкого давления имеет сильно неравновесный характер. Рождение заряженных частиц происходит
преимущественно в объемных процессах, а гибель на стенках разрядной камеры. Энергию электроны приобретают,
разгоняясь в электрическом поле, а теряют в упругих и неупругих столкновениях. Количественное описание этих процессов
возможно только на кинетическом уровне \cite{Zobnin}.

Как известно, существует один из подходов описания системы на кинетическом уровне при помощи уравнения Больцмана,
которое представляет собой интегродифференциальное уравнение, описывающее поведение разреженного газа.
Данное уравнение было выведено Людвигом Больцманом в 1872 г. Оно до сих пор остается основой кинетической теории
газов и оказывается плодотворным не только для исследования классических газов, которые имел в виду Больцман,
но - при соответствующем обращении - и для излучения переноса электронов в твердых телах и плазме \cite{Cherchin'yani}.
\begin{equation}{\partial f_e \over \partial t} + (\vec{v}, \vec{\Delta}_r) f_e + {e \over m_e} (\vec{E}, \vec{\Delta}_v) f_e = S_{coll} \end{equation}
где \math{S_{coll}}$ — интеграл столкновений, \math{f_e = f_e(\vec{r}, \vec{v}, t)}$ - функция распределения электронов

Для решения данного уравнения достаточно использовать двучленное приближение:
\begin{equation}f_e(\vec{r}, \vec{v}, t) = f_0(\vec{r}, \vec{v}, t) + { \vec{v} \over v} f_1(\vec{r}, \vec{v}, t)\end{equation}
где \math{f_1}$ отвечает за анизотропию.

Подставив это приближение в исходное уравнение и усреднив по направлениям скоростей, получим следующую систему уравнений:
\begin{equation}
 \begin{cases}
  \nu f_1 = - v \nabla f_0 - {eE_z \over m} {\partial f_o \over \partial v},
   \\
   {\partial f_0 \over \partial t} + {v \over 3} div(f_1) + {1 \over 3v^2} {\partial \over \partial v } ({ v^2 e E_z \over m} f_1 ) = S_0
 \end{cases}
\end{equation}
Данную систему иногда называют системой Давыдова-Эллиса [ССЫЛКА НА ИСТОЧНИК]. \math{S_0}$ —  интеграл столкновений,
отвечающий за упругие и неупругие электрон-атомные столкновения и электрон-электронные взаимодействия.
\begin{equation}
    S_0 = \sum_k \big{[} \sqrt{\epsilon + \epsilon_k} \nu_k (\epsilon + \epsilon_k)f(\epsilon + \epsilon_k) - \sqrt{\epsilon} \nu_k(\epsilon)f(\epsilon) \big{]}
\end{equation}
где суммирование проводится по всем верхним уровням, а \math{\nu_k(\epsilon) = \sigma_k(\epsilon)N_g\sqrt{\epsilon}}$,
\math{N_g}$ - концентрация атомов

Перейдя от скоростей к энергиям в системе Давыдова-Эллиса и подставив выражение для \math{f_1}$ в нижнее уравнение, получим:
\begin{equation}
    \begin{gathered}
        \sqrt{\epsilon} {\partial f \over \partial t } = \sqrt{2 \over m} \big{(} N_g S_0 + 2 {m \over m_{{Ne}_2}} N_g
        {\partial \over \partial \epsilon} \big{[} \sigma_t(\epsilon) \epsilon^2 f(\epsilon) \big{]} + \\
        + {e^2 E_z^2 \over 3 N_g} {\partial \over \partial \epsilon} \big{[} {\epsilon \over \sigma_t(\epsilon) }
        {\partial f \over \partial \epsilon} \big{]} \big{)}
    \end{gathered}
\end{equation}

Если предположить, что функция распределения электронов с течением времени выйдет на стационарный уровень,
то данную формулу можно использовать для построения разностной схемы.

Метод временной эволюции позволяет вычислить функцию распределения электронов из предыдущей формулы с помощью
следующей разностной схемы:

\begin{equation}
\begin{small}
    \begin{gathered}
        f^n(\epsilon) - f^{n-1}(\epsilon) = \Delta t \sqrt{2 \over m_e \epsilon} \Big{[} N_g \sum_k \big{[} (\epsilon +
        \epsilon_k)\sigma_k(\epsilon + \epsilon_k)f^{n-1} (\epsilon + \epsilon_k) - \epsilon \sigma_k(\epsilon)f^{n-1} (\epsilon) \big{]} + \\
        + {2m_e \over m_{{Ne}_2}} N_g {f^{n-1} (\epsilon + \Delta \epsilon) \sigma_t(\epsilon + \Delta \epsilon) (\epsilon + \Delta \epsilon)^2 - f^{n-1}
        (\epsilon - \Delta \epsilon) \sigma_t(\epsilon - \Delta \epsilon) (\epsilon - \Delta \epsilon)^2\over 2 \Delta \epsilon} + \\
         + {e^2 E_z^2 \over 3 N_g}  {1 \over \Delta \epsilon} \big{[} {(\epsilon + { \Delta \epsilon \over 2 }) (f^{n-1}(\epsilon +
         \Delta \epsilon) - f^{n-1}(\epsilon)) \over \Delta \epsilon \sigma_t(\epsilon + { \Delta \epsilon \over 2 })} -
         {(\epsilon - { \Delta \epsilon \over 2 }) (f^{n-1}(\epsilon) - f^{n-1}(\epsilon - \Delta \epsilon)) \over \Delta
         \epsilon \sigma_t(\epsilon - { \Delta \epsilon \over 2 })} \big{]} \Big{]}
    \end{gathered}
\end{small}
\end{equation}
где \math{f^n}$~--~n-конфигурация ФРЭ во времени, n~--~это порядковый номер шага по времени;
\math{\epsilon}$~--~энергия в~эВ; \math{\Delta \epsilon}$~--~шаг по энергии в~эВ;
\math{\sigma_t (\epsilon)}$~--~транспортное сечение упругого рассеяния;
\math{\sigma_k (\epsilon)}$~--~сечение неупругих столкновений для k-уровня;
\math{N_g}$~--~концентрация атомов Ne; \math{m_{{Ne}_2}}$~--~масса молекулы Ne;
\math{m_e}$~--~масса электрона;
\math{E_z}$~--~осевое электрическое поле, которое задается параметрически в диапазоне \math{[1, 10]}$~В/см с шагом 0.1~В/см;
\math{e}$~--~заряд электрона

Использовались следующие граничные условия:
\begin{equation}
    \begin{cases}
        {d f \over d \epsilon}(0) = 0
        \\
        f(\infty) = 0
    \end{cases}
    \sim~~~
    \begin{cases}
        f_{n}^0 = f_{n}^{1}
        \\
        f_{n}^K = 0
    \end{cases}
\end{equation}
K~--~количество шагов по энергии за одну итерацию по времени, в рамках данной задачи 50~эВ можно считать уже бесконечно большой.

\section{Решение уравнения Больцмана и результаты}
\begin{figure}[t]
  \centering
  \includegraphics[width=15cm]{figures/fig14}
  \caption{Зависимость расчетной функции распределения электронов от энергии и осевого электрического поля,
  заданного параметрически для некоторых значений.}
  \label{fig:fig14}
\end{figure}

В качестве начальных условий первого вычисления ФРЭ (при \math{E = 1}$~В/см) удобнее всего выбрать распределение
Больцмана, поскольку оно близко к итоговому решению. Удобство заключается в скорости сходимости алгоритма:
чем приближеннее возьмем начальное условие, тем быстрее наступит стационарный уровень.
Затем для расчета ФРЭ для следующего поля лучше всего использовать в качестве начальных условий решение
от предыдущего поля (см.~рис~\ref{fig:fig14}).

Особый интерес в исследовании ФРЭ представляет зависимость в логарифмическом масштабе, поскольку заметить различия
между расчетным распределением и распределением Больцмана на глаз практически невозможно. Распределение Больцмана
в логарифмическом масштабе представляет собой линейную зависимость в отличие от расчетной (см.~рис~\ref{fig:fig15}).
\begin{figure}[t]
    \begin{center}
         \subfloat[\label{sub:fig15a}]{
           \includegraphics[width=0.49\textwidth]{figures/fig15a}
         }
         \subfloat[\label{sub:fig15b}]{
           \includegraphics[width=0.49\textwidth]{figures/fig15b}
         }
         \caption{Зависимость расчетной функции распределения электронов от энергии и осевого электрического поля,
                  заданного параметрически для некоторых значений, в логарифмическом масштабе:
                  \pt(a) в диапазоне [1, 15]~эВ, \pt(b) в диапазоне [10, 30]~эВ
         }
         \label{fig:fig15}
    \end{center}
\end{figure}

Из данного графика видно, что расчетная функция распределения электронов принимает двухтемпературный вид.
Это связано с тем, что электроны, имеющие энергию выше пороговой (\math{E_{thr}}$) участвуют в неупругих
столкновениях с атомами.

Поскольку в логарифмическом масштабе функция распределения электронов представляет собой два четко выраженных
линейных участка (двухтемпературный вид), это позволяет рассматривать эти участки независимо друг от друга
со своими электронными температурами:
\begin{equation}
    f_e = e^{-{\epsilon \over T_e}} \Rightarrow ln(f_e) = - {\epsilon \over T_e}
\end{equation}

Зная электронную температуру невозмущенного пылевыми частицами газового разряда, можно определить электронную
температуру с пылевыми частицами из соотношения логарифмов функций распределения энергий:
\begin{equation}
    T_{e,2} = {ln(f_{e,1}) \over ln(f_{e,2})} T_{e,1}
\end{equation}
Для разрешения данного соотношения необходимо понять какие из функций распределения электронов нужно использовать.

Скорость заселения верхних уровней определяется следующим выражением:
\begin{equation}
    X_{exc} (T_e) = \sqrt{2 \over m_e} \int_{E_{th}}^{\infty} \sigma(\epsilon)f_e(\epsilon)\sqrt{\epsilon}d\epsilon
\end{equation}
где \math{\sigma (\epsilon)}$~--~сечение (вероятность перехода) в зависимости от энергии
\math{f_e (\epsilon)}$~--~функция распределения электронов по энергиям

\begin{figure}[t]
  \centering
  \includegraphics[width=12cm]{figures/fig16}
  \caption{Зависимость ...... .}
  \label{fig:fig16}
\end{figure}

В рамках одного и того же переходного процесса справедливы следующие рассуждения: интенсивность
спектральной линии пропорциональна заселенности верхнего уровня данного перехода, которая в свою очередь
пропорциональна скорости заселения верхнего уровня, т. е.:
\begin{equation}
I(T_e) \sim N^*(T_e) \sim X_{exc}(T_e) \sim  \int_{E_{th}}^{\infty} \sigma(\epsilon)f_e(\epsilon)\sqrt{\epsilon}d\epsilon
\end{equation}

Перейдя к отношению интенсивностей:
\begin{equation}
{{I_2} \over {I_1}}= {{\int_{E_{th}}^{\infty} \sigma(\epsilon)f_{e,2}(\epsilon)\sqrt{\epsilon}d\epsilon} \over
{\int_{E_{th}}^{\infty} \sigma(\epsilon)f_{e,1}(\epsilon)\sqrt{\epsilon}d\epsilon}}
\end{equation}

Из последнего выражения можно сделать вывод, что отношение интенсивностей задается функцией распределения электронов,
а также энергией верхнего уровня. Узнав значение функции распределения электронов для различных значений энергий в
зависимости от кинетических параметров, а также выбрав  фиксированную линию, можно на основе измеренных значений
отношений интенсивностей узнать относительное изменение кинетических параметров системы, таких как электронная
температура и осевое электрическое поле.

Также хочется подчеркнуть, что энергия верхнего уровня влияет на значение отношения интенсивностей, что мы и наблюдаем в эксперименте.

Для невозмущенного состояния исследуемого газового разряда осевое поле было получено из работ […]
и составляет 2.2~В/см, что дает возможность определить ФРЭ \math{f_{e,1}}$. Были перебраны все расчетные функции
распределения электронов и выбрана та, которая наиболее близко приближает экспериментальные отношения
интенсивностей (см.~рис~\ref{fig:fig16}).

Были получены следующие значения:

