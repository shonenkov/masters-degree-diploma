\Introduction

Диагностика плазмы является основой для выбора теоретических моделей и интерпретации полученных данных.
Главные диагностируемые параметры в не намагниченной плазме - концентрация ионов \math{n_i}$
и электронов \math{n_e}$, температура ионов \math{T_i}$ и электронов \math{T_e}$,
функция распределения электронов \math{f_e}$, а также распределение
пространственно-электрического потенциала \phi(r)$. Зная последнее, можно получить распределение напряженностей
электрических полей \math{E = \phi(r)}$.

Существует целый спектр различных методов диагностики плазмы - электрические (зондовые), оптические, спектральные,
корпускулярные, микроволновые [?...ССЫЛКА НА ИСТОЧНИК...?]. Как правило, плазма существенно неоднородная, а измерения содержат
существенные ошибки. Наиболее достоверными являются параметры плазмы, измеренные двумя независимыми методами.
Использование комбинаций различных методов позволяют получать данные, недоступные каждому методу в отдельности.
Например, спектральные методы позволяют экстраполировать в пространстве данные зондовых измерений.

Несмотря на свой почтенный возраст, зондовая диагностика остается наиболее достоверным базовым видом диагностики плазмы.
В общем случае из полученной зондовой вольт-амперной характеристики можно извлечь все основные параметры плазмы.
Основным недостатком зондового метода является его инвазивность и необходимость обустройства специальных фланцев для ввода зондов.
Что касается зондовой диагностики пылевой плазмы, то она мало перспективна ввиду того, что зонд крайне сильно возмущает пылевое облако.
В настоящее время на Международной космической станции (МКС) находится российско-европейская научная аппаратура
«Плазменный~кристалл~-~4» (НА~«ПК-4») для изучения фундаментальных свойств пылевой плазмы в положительном столбе
газового разряда низкого давления. Одной из задач этого эксперимента является изучение влияния пылевой компоненты
на спектр излучения положительного столба и определение по этому изменению спектра изменение параметров плазмы -
в первую очередь изменения электронной температуры. Полноценная интерпретация полученных спектральных данных требует
составления и самосогласованного решения кинетического уравнения Больцмана в нелокальном приближении для положительного
столба с пылевой компонентой, что является сложной задачей.

Целью данной работы является исследование влияния протяженного пылевого облака на спектральные характеристики
положительного столба газового разряда и определение по данным характеристикам изменения электронной температуры
в облаке.

В связи с этим, был проведен обзор экспериментальных спектральных данных, полученных на научной аппаратуре «ПК-4» за
время ее эксплуатации, выбор наиболее удачных с точки зрения поставленной задачи экспериментов, обработка спектров
и их интерпретация в рамках столкновительно-радиационной модели.
